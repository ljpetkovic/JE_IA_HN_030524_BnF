\begin{frame}{Fonds Charcot \href{https://patrimoine.sorbonne-universite.fr/collection/Fonds-Charcot}{\textcolor{yellow}{en ligne}}}
\begin{block}{SorbonNum\\
\footnotesize{Bibliothèque de Sorbonne Université (\textsc{BSU})}}
201 documents XML OCRisés (sans post-correction)
\end{block}
%\begin{itemize}
%    \item \textrm{Charcot} : textes rédigés par Charcot
%    \item \textrm{Autres} : textes rédigés par les membres de son réseau scientifique
%\end{itemize}
\begin{table}[!ht]
    \centering
    \begin{tabular}{|c|r|r|}
    \hline 
    \rowcolor{yellow!30}
       Corpus & \multicolumn{1}{c|}{Nb de docs} & \multicolumn{1}{c|}{Nb de tokens} \\
       \hline
      \begin{tabular}[c]{@{}c@{}}\textrm{Charcot}\\ \scriptsize{textes rédigés par Charcot}\end{tabular}  & 68 & 12 190 649 (38,12\%) \\
       \hline
       \begin{tabular}[c]{@{}c@{}}\textrm{Autres}\\ \scriptsize{textes rédigés par les membres} \vspace{-0.15cm} \\ \scriptsize{de son réseau scientifique}\end{tabular}    & 133 & 19 788 830 (61,88\%) \\
       \hline\hline
       \textbf{Total} & \textbf{201} & \textbf{31 979 479} (100\%)\\
       \hline
    \end{tabular}
    \caption{Répartition du fonds Charcot selon les auteurs.
%    \footnote{\tiny{\url{https://patrimoine.sorbonne-universite.fr/collection/Fonds-Charcot}}}
}
    \label{tab:my_label}
\end{table}
\end{frame}

\begin{frame}{Corpus Charcot \href{https://obtic.huma-num.fr/obvie/charcot/?view=corpus}{\textcolor{yellow}{en ligne}}}
Corpus Charcot accessible sur la plateforme \textsc{OBVIE} \hfill {\small\citep{alrahabi2022obvie}}
\begin{itemize}
\item fouille avancée des corpus en \textsc{XML-TEI}
\item textes similaires : mots fréquents / en commun, noms cités
\end{itemize}
%\danger impossible de quantifier l'importance des MWEs
\begin{figure}[!h]
    \centering
\includegraphics[width=90mm,scale=0.5]{pic/doc_sim.png}
    \caption{Points similaires entre un ouvrage de Charcot et celui de de la Tourette.}
%    \caption{Distribution des fréquences des tokens avec la frise chronologique pour ceux constituant l'expression \textit{bulbe rachidien} (issus des corpus \og{}Charcot\fg{} et \og{}Autres\fg{}).}
    \label{fig:my_label}
\end{figure}
% citations directes (\cite{manjavacas2019})
\end{frame}

\begin{frame}{Mesurer le degré d'intertextualité}
Mesurer informatiquement l'impact de Charcot sur son réseau \\$\rightarrow$ intertextualité uni-directionnelle
\begin{figure}[!h]
    \centering
\includegraphics[width=100mm,scale=0.5]{pic/charcot_intertextualite.png}
    \caption{Opérationnalisation de l'impact de Charcot sur ses élèves.}
    \label{fig:my_label}
\end{figure}
\end{frame}

%\begin{frame}{Première analyse}
%\textbf{OBVIE}\footnote{\url{https://obtic.huma-num.fr/obvie/}}
%\begin{itemize}
%    \item moteur de recherche permettant la fouille avancée des corpus en \textsc{XML-TEI}
%    \item identification des substantifs les plus importants de chaque corpus 
%    \begin{itemize}
%        \item fréquences brutes, mesures \textsc{TF-IDF}, \textsc{BM25}, $\chi^2$, Test Gamma
%    \end{itemize}
%    \item repérage des textes similaires par ordre de pertinence à partir des termes en commun
%\end{itemize}
    
%\end{frame}


%\begin{frame}{Deuxième analyse}
%    \textbf{TextPair}\footnote{\url{https://artfl-project.uchicago.edu/text-pair}}
%    \begin{itemize}
%        \item alignement des séquences similaires dans les deux corpus
%        \item génère une liste de passages similaires pour chaque texte
%        \item séquences de mots qui se chevauchent (trigrammes de mots)
%        \item comparer ces résultats avec ceux de séquences dans d’autres textes
%    \end{itemize}
%\end{frame}
%
%\begin{frame}{Deuxième analyse -- TextPair\footnote{\url{https://anomander.uchicago.edu/text-pair/charcot2autres/}}}
%\danger nombre de
%résultats assez conséquent -- filtrage nécessaire
%    \begin{figure}[!ht]
%        \centering
%        \includegraphics[width=90mm,scale=0.5]{pic/textpair.png}
%        \caption{Alignement et comparaison des textes de
%Charcot à celui de Georges Gilles de la Tourette (le seul
%résultat) en lançant la requête \textit{sclérose latérale
%amyotrophique}.}
%        \label{fig:enter-label}
%    \end{figure}
% \end{frame}
