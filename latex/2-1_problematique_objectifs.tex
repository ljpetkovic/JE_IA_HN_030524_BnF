\begin{frame}{Impact de Charcot sur sa discipline et au-delà}
\centering
\textbf{(Quelques) collaborateurs et élèves}\\
{\small \og{}réseau scientifique\fg{}}
    \begin{table}[!ht]
        \centering
        \small
        \begin{tabular}{l r}
           Sigmund \textsc{Freud} (1856-1939)  & théorie psychanalytique \\
            Gilles \textsc{de la Tourette} (1857-1932) & syndrôme de Tourette \\
            Joseph \textsc{Babinski} (1857-1904) & pithiatisme, signe de Babinski \\
%           Pierre \textsc{Janet} (1859-1947) & psychopathologie
%            dissociation, sous-conscient 
        \end{tabular}
        \begin{flushright}
%        \footnotesize\citep{bogousslavsky2020}
        \footnotesize\citep{BROUSSOLLE2012301}
        \end{flushright}
        % \caption{Caption}
        \label{tab:my_label}
    \end{table}
\medskip
\textbf{(Quelques) écrivains naturalistes français et européens} 
\begin{itemize}
\centering
\small \item références à Charcot et aux descriptions de crises hystériques
\end{itemize}
\begin{table}[!ht]
    \centering
    \small
    \begin{tabular}{l r}
        Émile \textsc{Zola} (1840–1902)  & \textit{Lourdes} \\
        Léon \textsc{Tolstoï} (1828–1910) & \textit{La Sonate à Kreutzer} \\
        Luigi \textsc{Capuana} (1839–1915) & \textit{La Torture}
%        Bjørnstjerne \textsc{Bjørnson} (1832–1910) & \textit{Over Ævne}
    \end{tabular}
            \begin{flushright}
        \footnotesize\citep{koehler2013charcot}
        \end{flushright}
    % \caption{Caption}
    \label{tab:my_label}
\end{table}

\end{frame}

